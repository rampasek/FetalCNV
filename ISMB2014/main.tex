\pdfoutput=1
\documentclass{bioinfo}
\copyrightyear{2014}
\pubyear{2014}

%\usepackage{dblfloatfix} 
\usepackage{natbib}
\usepackage{amssymb}
\usepackage{graphicx}
\usepackage{amsmath}
\usepackage{subfigure}
\usepackage{listings}
\usepackage{url}
%\usepackage{cite}
\usepackage{relsize}
\usepackage{todonotes}
\usepackage{multirow}


\newcommand{\vmu}{\mbox{\boldmath${\mu}$} }
\newcommand{\vSigma}{\mathbf{\Sigma} }
\renewcommand{\(}{\left(}
\renewcommand{\)}{\right)}
\renewcommand{\{}{\left\lbrace }
\renewcommand{\}}{\right\rbrace }
\newcommand{\N}{\mathcal{N}}
\newcommand\WRV{\mathit{WRV}}
\newcommand\TP{\mathit{TP}}
\newcommand\FN{\mathit{FN}}
\newcommand\IP{\mathit{IP}}
\newcommand\PP{\mathit{PP}}
\newcommand\A{\texttt{A}}
\newcommand\C{\texttt{C}}
\newcommand\G{\texttt{G}}
\newcommand\T{\texttt{T}}
\newcommand\f{\mathrm{f}}
\newcommand\m{\mathrm{m}}
\newcommand\p{\mathrm{p}}

% nice tilde
\newcommand{\ntilde}{\raise.17ex\hbox{$\scriptstyle\mathtt{\sim}$}}

\def\inst#1{${}^{#1}$}

\begin{document}

\title[Probabilistic Method for Detecting CNVs in a Fetal Genome using Maternal Plasma Sequencing]{Probabilistic Method for Detecting Copy Number Variation in a Fetal Genome using Maternal Plasma Sequencing}
\author[Ramp\'a\v{s}ek et al.]{
    Ladislav Ramp\'a\v{s}ek\inst{1},
    Aryan Arbabi\inst{1}, and
    Michael Brudno\inst{1,2,3}\footnote{To whom correspondence should be addressed}}

\urldef{\mailsa}\path|{rampasek, arbabi, brudno}@cs.toronto.edu|
\address{
\inst{1}
    Department of Computer Science, University of Toronto, Toronto M5S 2E4, Canada\\
\inst{2}
	Centre for Computational Medicine, Hospital for Sick Children, Toronto M5G 1L7, Canada\\
\inst{3}
	Genetics and Genome Biology, Hospital for Sick Children, Toronto M5G 1L7, Canada
}

\history{Received on XXXXX; revised on XXXXX; accepted on XXXXX}
\editor{Associate Editor: XXXXXXX}


\maketitle

\begin{abstract}
\section{Motivation:}
The past 6 years have seen the development of methodologies to identify genomic variation within a fetus through the non-invasive sequencing of maternal blood plasma. These methods are based on the observation that maternal plasma contains a fraction of DNA (typically 5-15\%) originating from the fetus, and such methodologies have already been used for the detection of whole-chromosome events (aneuploidies), and to a more limited extent for smaller (typically several megabases long) Copy Number Variants (CNVs).

\section{Results:}
Here we present a probabilistic method for non-invasive analysis of \textit{de novo} CNVs in fetal genome based on maternal plasma sequencing. Our novel method combines three types of information within a unified Hidden Markov Model: the imbalance of allelic ratios at SNP positions, the use of parental genotypes to phase nearby SNPs, and depth of coverage to better differentiate between various types of CNVs and improve precision. Our simulation results, based on \emph{in silico} introduction of novel CNVs into plasma samples with 13\% fetal DNA concentration, demonstrate a sensitivity of 90\% for CNVs $>$400 kilobases (with 13 calls in an unaffected genome), and 40\% for 50-400kb CNVs (with 108 calls in an unaffected genome).

\section{Availability:}
Implementation of our model and data simulation method is available at \path|http://github.com/rampasek/FetalCNV|

\section{Contact:}
\path|{rampasek, arbabi, brudno}@cs.toronto.edu| 
%\keywords{non-invasive, prenatal, maternal plasma, CNV}
\end{abstract}

\input intro
\input alg
\input res
\input dis

\subsection*{Acknowledgements}
We would like to thank Orion Buske and Misko Dzamba for their helpful comments and discussions.

\bibliographystyle{natbib}
\bibliography{main}

\end{document}
