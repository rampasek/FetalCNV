\section{Discussion}

In this manuscript we introduce a novel probabilistic method for the identification of \textit{de novo} Copy Number Variants from maternal blood plasma sequencing. Our method combines three types of data: allelic ratios, reflecting the changes in the expected observations of various alleles at SNP positions in the presence of the CNV; phasing information, allowing for the combining of allelic ratios across multiple SNP positions, thus improving the signal-to-noise ratio, and depth of coverage information reflecting the change in expected sequencing depth in the presence of the CNV. We apply the resulting method to simulated sequencing data, demonstrating promising results for CNVs $>400$ kilobases in length, and especially for CNVs of paternal origin. 
Simultaneously, we believe our method can be further improved in several ways. First, our approach of modelling the depth of coverage information as a prior, within small bins is likely suboptimal. Espcially because the method is searching for larger CNVs, using larger bins would be advantageous; however in this case the observations of coverage at adjacent SNPs would no longer be independent,  and thus not properly modelled as an HMM. We believe a more expressive model that is able to model such interactions between coverage terms would improve upon the current results. Secondly, our method does not directly model potential inherited CNVs in the father (maternally inherited CNVs are modelled through the use of maternal priors at each position). Explicitly pre-computing and utilizing information about these inherited CNVs is likely to reduce the false positive rate of ours and related methods.

